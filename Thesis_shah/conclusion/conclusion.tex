\chapter{Discussions and Conclusions}
\section{Discussion}

The most obvious area where further work may be performed, especially for the research covered in chapters 3 and 4, is the utilisation of models generated from the latest methods.  The field of protein structure prediction is developing at an incredible pace with new methods becoming available on an almost daily basis.  In terms of the DedA fold, we believe we have captured this accurately and the topology has since been experimentally verified \cite{okawa2021evolution,scarsbrook2021topological}. The question still remains, however, whether the DedA models are stable within a membrane.  It is clear that having the re-entrant loop exposed to lipid bi-layer would not be stable. Experimental evidence is available that these proteins exist as homodimers \cite{scarsbrook2021topological}.  The homodimers could plausibly shield the re-entrant loops from the hydrophobic environment of the membrane. It is now possible using AlphaFold2 or specialist oligomerisation software such as AlphaPulldown \cite{yu2023alphapulldown} to build potential homodimers.  The stability of these homodimer models when placed in a lipid bilayer could be tested using molecular dynamic simulations.  Additionally, further work needs to be carried out to determine what the substrate is for both DedA proteins and Atg9 as well as to provide validation that Oca2 transports a dicarboxylate such as citrate.  Bioinformatic methods for making ligand predictions based on deep learning methods are beginning to emerge \cite{kandel2021puresnet} and these could shed light on what these putative transporters are actually translocating.  Finally, the methodology for mining databases for specific membrane protein sub-structures could have been made more efficient through the inclusion of additional structural attributes. For example, identifying the re-entrant/transmembrane helix motifs could have been made more efficient and sensitive if the transmembrane subset of the AlphaFold database had first been processed with PDBTM's TMDET \cite{Kozma2012} algorithm. TMDET would have provided re-entrant loop annotation to the library of the transmembrane structures. This would have eliminated the need to manually place structures into the membrane using OPM \cite{Lomize2012} and visually inspect the membrane placed structures to identify hits.  An annotated transmembrane AlphaFold database could also be used to perform a full census of proteins that possess re-entrant loops as well as investigate other attributes of these regions such as residue composition or fold patterns that re-entrant loops form with adjacent structures in three dimensional space.

This study employed searches of structural databases for specific re-entrant structural motifs.  Another key structural feature of the target proteins studied in Chapters 3 and 4 is the tandem repeat.  Additional work could be carried out to mine for this interesting feature. Repeat proteins are commonly found across all domains of life and harbour a wide range of functions \cite{andrade2001protein}. The origin of these repeats derives from gene duplication events \cite{heringa1998detection} resulting from slipped-strand mispairing during DNA replication \cite{paques1998expansions} or through replication arising during the repairing of double strand DNA breaks \cite{paques1998expansions}. The length of the repeating sequence unit varies considerably \cite{heringa1998detection,tompa2003intrinsically} and can be used in conjunction with their tertiary structure to group repeat proteins into five classes \cite{kajava2006beta} a classification also adopted by the RepeatsDB; crystalline aggregates, fibrous repeat, elongated repeat, closed repeat, beads-on-a-string \cite{di2014repeatsdb}.  Many families of repeat proteins remain structurally uncharacterised by experimental methods, yet they are particularly favourable subjects for modelling: there is an expectation that each sequence repeat should adopt the same structural configuration, a rule that provides a form of internal validation.   



\section{Conclusions}
The past three years has seen an exponential improvement in protein structure prediction owing to the implementation of deep learning methods into the model building algorithms.  Indeed, high accuracy protein modelling has historically been associated with template based modelling where back in CASP2 \cite{dixon1997evaluation}, if a template was available, the best models were obtaining GDT\_TS scores of above 80\%; without a template the output models were considered random.  Over the years some small improvements were made to the accuracy of free modelling methods, especially for smaller proteins, where fragments assembly algorithms such as Rosetta \cite{baker2001protein} were implemented.  By CASP11 \cite{monastyrskyy2016new} some further improvements in ab initio modelling were observed where contact predictions had been incorporated into the model building process.  It wasn't until CASP13 \cite{cheng2019estimation}, with the use of deep learning for distance predictions by Deepmind's AlphaFold \cite{senior2019protein}, that models without a templates were scoring as well as template-based models.  Furthermore, with the introduction of AlphaFold2 \cite{Jumper2021} in CASP14, targets with or without available templates, were achieving GDT\_TS scores of above 90\%; AlphaFold2 had been re-engineered.  Along side other significant modifications there was the introduction of a key mechanism (attention algorithm) to reverse the tendency of the neural network to prefer models with more secondary structure \cite{Jumper2021}. 

This PhD demonstrated how contacts could be used to aid the model building process to output an accurate biologically relevant fold. Subsequently, the evolution of utilising MSAs for the prediction of contacts to the prediction of distances has been invaluable for highly accurate protein prediction.  Historically contact prediction and model construction were performed as independent stages in the model generation process. The new generation of model building algorithms combine these steps, predicting distances rather than binary contacts.  These distances are used as constraints which are refined as decoy models are built.  The availability of highly accurate protein models will be useful in the directing of experimental investigations where experimental structures are not available.  Indeed our models of the DedA fold were utilised in this way \cite{tiwari2021klebsiella,chen2022scramblases,scarsbrook2021topological,hama2022regulation}.  Our work over the course of this PhD has introduced novel tools that can be utilised in prediction data analysis (ConPlot) as well as novel methods that can be employed to enhance protein model building protocols (utilisation of homology models to sample conformations).  The availability of highly accurate models may also be used, bioinformatically coupled with mining of curated sequence and structure databases, for inference of function and molecular mechanism, as demonstrated with our work on Oca2. We have also demonstrated that contact analysis still has a role to play in modern structural bioinformatics; in model validation and conformation deciphering. Molecular dynamic simulations were not taken advantage of during this PhD; great strides have also been made in this area as a result of increased computing power utilising GPUs as well as improved energy functions.  In terms of studying membrane proteins specifically, all of these developments give rise to the opportunity to understand how they interact with other proteins, substrates, and membranes of various types. Deep learning methods have brought about a new era in protein modelling with the outlook for membrane protein modelling in the immediate future looking very bright indeed. 



